\documentclass[conference]{IEEEtran}
\IEEEoverridecommandlockouts
% The preceding line is only needed to identify funding in the first footnote. If that is unneeded, please comment it out.
\usepackage{cite}
\usepackage{amsmath,amssymb,amsfonts}
\usepackage{algorithmic}
\usepackage{graphicx}
\usepackage{textcomp}
\usepackage{xcolor}

\providecommand{\tightlist}{%
  \setlength{\itemsep}{0pt}\setlength{\parskip}{0pt}}


\def\BibTeX{{\rm B\kern-.05em{\sc i\kern-.025em b}\kern-.08em
    T\kern-.1667em\lower.7ex\hbox{E}\kern-.125emX}}
\begin{document}

\title{Conference Paper Title\\
\thanks{}
}

\author{\IEEEauthorblockN{1\textsuperscript{st} Given Name Surname}
\IEEEauthorblockA{\textit{dept. name of organization (of Aff.)} \\
\textit{name of organization (of Aff.)}\\
City, Country \\
email address or ORCID}
\and
\IEEEauthorblockN{2\textsuperscript{nd} Given Name Surname}
\IEEEauthorblockA{\textit{dept. name of organization (of Aff.)} \\
\textit{name of organization (of Aff.)}\\
City, Country \\
email address or ORCID}
\and
\IEEEauthorblockN{3\textsuperscript{rd} Given Name Surname}
\IEEEauthorblockA{\textit{dept. name of organization (of Aff.)} \\
\textit{name of organization (of Aff.)}\\
City, Country \\
email address or ORCID}
\and
\IEEEauthorblockN{4\textsuperscript{th} Given Name Surname}
\IEEEauthorblockA{\textit{dept. name of organization (of Aff.)} \\
\textit{name of organization (of Aff.)}\\
City, Country \\
email address or ORCID}
\and
\IEEEauthorblockN{5\textsuperscript{th} Given Name Surname}
\IEEEauthorblockA{\textit{dept. name of organization (of Aff.)} \\
\textit{name of organization (of Aff.)}\\
City, Country \\
email address or ORCID}
\and
\IEEEauthorblockN{6\textsuperscript{th} Given Name Surname}
\IEEEauthorblockA{\textit{dept. name of organization (of Aff.)} \\
\textit{name of organization (of Aff.)}\\
City, Country \\
email address or ORCID}
}

\maketitle

\begin{abstract}
This document is a model and instructions for LaTeX. This and the
IEEEtran.cls file define the components of your paper {[}title, text,
heads, etc.{]}. *CRITICAL: Do Not Use Symbols, Special Characters,
Footnotes, or Math in Paper Title or Abstract.
\end{abstract}

\begin{IEEEkeywords}
onetwothree
\end{IEEEkeywords}

\section{Introduction}\label{introduction}

This document is a model and instructions for \LaTeX. Please observe the
conference page limits.

\section{Ease of Use}\label{ease-of-use}

\subsection{Maintaining the Integrity of the
Specifications}\label{maintaining-the-integrity-of-the-specifications}

The IEEEtran class file is used to format your paper and style the text.
All margins, column widths, line spaces, and text fonts are prescribed;
please do not alter them. You may note peculiarities. For example, the
head margin measures proportionately more than is customary. This
measurement and others are deliberate, using specifications that
anticipate your paper as one part of the entire proceedings, and not as
an independent document. Please do not revise any of the current
designations.

\section{Prepare Your Paper Before
Styling}\label{prepare-your-paper-before-styling}

Before you begin to format your paper, first write and save the content
as a separate text file. Complete all content and organizational editing
before formatting. Please note sections \ref{AA}--\ref{SCM} below for
more information on proofreading, spelling and grammar.

Keep your text and graphic files separate until after the text has been
formatted and styled. Do not number text heads---\{\LaTeX\} will do that
for you.

\subsection{Abbreviations and
Acronyms}\label{abbreviations-and-acronyms}

Define abbreviations and acronyms the first time they are used in the
text, even after they have been defined in the abstract. Abbreviations
such as IEEE, SI, MKS, CGS, ac, dc, and rms do not have to be defined.
Do not use abbreviations in the title or heads unless they are
unavoidable.

\subsection{Units}\label{units}

\begin{itemize}
\tightlist
\item
  Use either SI (MKS) or CGS as primary units. (SI units are
  encouraged.) English units may be used as secondary units (in
  parentheses). An exception would be the use of English units as
  identifiers in trade, such as \texttt{3.5-inch\ disk\ drive}.
\item
  Avoid combining SI and CGS units, such as current in amperes and
  magnetic field in oersteds. This often leads to confusion because
  equations do not balance dimensionally. If you must use mixed units,
  clearly state the units for each quantity that you use in an equation.
\item
  Do not mix complete spellings and abbreviations of units:
  \texttt{Wb/m\textbackslash{}textsuperscript\{2\}} or
  \texttt{webers\ per\ square\ meter}, not
  \texttt{webers/m\textbackslash{}textsuperscript\{2\}}. Spell out units
  when they appear in text: \texttt{.\ .\ .\ a\ few\ henries}, not
  \texttt{.\ .\ .\ a\ few\ H}.
\item
  Use a zero before decimal points: \texttt{0.25}, not \texttt{.25}. Use
  \texttt{cm\textbackslash{}textsuperscript\{3\}}, not \texttt{cc}.)
\end{itemize}

\section{Footnotes}\label{footnotes}

Example of footnote\footnote{A footnote example}. Lorem ipsum dolor sit
amet, consectetur adipisicing elit, sed do eiusmod tempor incididunt ut
labore et dolore magna aliqua. Ut enim ad minim veniam, quis nostrud
exercitation ullamco laboris nisi ut aliquip ex ea commodo consequat.
Duis aute irure dolor in reprehenderit in voluptate velit esse cillum
dolore eu fugiat nulla pariatur. Excepteur sint occaecat cupidatat non
proident, sunt in culpa qui officia deserunt mollit anim id est laborum.

\section{Cites}\label{cites}

Zotero + Better BibTex. All cites are on the file bibliography.bib. This
is a \cite{djangoproject}.

\begin{table}[htbp]
\caption{Table Type Styles}
\begin{center}
\begin{tabular}{|c|c|c|c|}
\hline
\textbf{Table}&\multicolumn{3}{|c|}{\textbf{Table Column Head}} \\
\cline{2-4} 
\textbf{Head} & \textbf{\textit{Table column subhead}}& \textbf{\textit{Subhead}}& \textbf{\textit{Subhead}} \\
\hline
copy& More table copy$^{\mathrm{a}}$& &  \\
\hline
\multicolumn{4}{l}{$^{\mathrm{a}}$Sample of a Table footnote.}
\end{tabular}
\label{tab1}
\end{center}
\end{table}

\begin{figure}
\hypertarget{fig1}{%
\centering
\includegraphics{img/fig1.png}
\caption{Example of a figure caption.}\label{fig1}
}
\end{figure}

\begin{figure}[htbp]
\centerline{\includegraphics{img/fig1.png}}
\caption{Example of a figure caption.}
\label{fig}
\end{figure}

\section{Conclusion}\label{conclusion}

Lorem ipsum dolor sit amet, consetetur sadipscing elitr, sed diam nonumy
eirmod tempor invidunt ut labore et dolore magna aliquyam erat, sed diam
voluptua. At vero eos et accusam et justo duo dolores et ea rebum. Stet
clita kasd gubergren, no sea takimata sanctus est Lorem ipsum dolor sit
amet.


\begin{thebibliography}{00}
\bibitem{djangoproject} G. Eason, B. Noble, and I. N. Sneddon, ``On certain integrals of Lipschitz-Hankel type involving products of Bessel functions,'' Phil. Trans. Roy. Soc. London, vol. A247, pp. 529--551, April 1955.
\end{thebibliography}

\end{document}
